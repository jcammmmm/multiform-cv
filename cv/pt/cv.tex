%%%%%%%%%%%%%%%%%%%%%%%%%%%%%%%%%%%%%%%
% Deedy - One Page Two Column Resume
% LaTeX Template
% Version 1.8 (13/04/2022)
%
% Original author:
% Debarghya Das ( )
% 
% Devs. and mods.
% Juan Camilo Florez
%
% Original repository:
% https://github.com/deedydas/Deedy-Resume
%
% IMPORTANT: THIS TEMPLATE NEEDS TO BE COMPILED WITH XeLaTeX
%
% This template uses several fonts not included with Windows/Linux by
% default. If you get compilation errors saying a font is missing, find the line
% on which the font is used and either change it to a font included with your
% operating system or comment the line out to use the default font.
% 
%%%%%%%%%%%%%%%%%%%%%%%%%%%%%%%%%%%%%%
% 
% TODO:
% 1. Integrate biber/bibtex for article citation under publications.
% 2. Figure out a smoother way for the document to flow onto the next page.
% 3. Add styling information for a "Projects/Hacks" section.
% 4. Add location/address information
% 5. Merge OpenFont and MacFonts as a single sty with options.
% 
%%%%%%%%%%%%%%%%%%%%%%%%%%%%%%%%%%%%%%
%
% CHANGELOG:
% v1.1:
% 1. Fixed several compilation bugs with \renewcommand
% 2. Got Open-source fonts (Windows/Linux support)
% 3. Added Last Updated
% 4. Move Title styling into .sty
% 5. Commented .sty file.
%
%%%%%%%%%%%%%%%%%%%%%%%%%%%%%%%%%%%%%%%
%
% Known Issues:
% 1. Overflows onto second page if any column's contents are more than the
% vertical limit
% 2. Hacky space on the first bullet point on the second column.
%
%%%%%%%%%%%%%%%%%%%%%%%%%%%%%%%%%%%%%%


\documentclass[]{CV-JuanCamiloFlorez}
\usepackage{fancyhdr}
 
\pagestyle{fancy}
\fancyhf{}
 
\begin{document}

%%%%%%%%%%%%%%%%%%%%%%%%%%%%%%%%%%%%%%
%
%     LAST UPDATED DATE
%
%%%%%%%%%%%%%%%%%%%%%%%%%%%%%%%%%%%%%%
\lastupdated

%%%%%%%%%%%%%%%%%%%%%%%%%%%%%%%%%%%%%%
%
%     TITLE NAME
%
%%%%%%%%%%%%%%%%%%%%%%%%%%%%%%%%%%%%%%
\namesection{Juan Camilo}{Flórez Vanegas}{ \urlstyle{same}
    \href{mailto:jcflorezv@unal.edu.co}{jcflorezv@unal.edu.co} 
    | \href{https://jcamilo.co}{https://jcamilo.co}
    | +57 322 326 67 29
}

%%%%%%%%%%%%%%%%%%%%%%%%%%%%%%%%%%%%%%
%
%     COLUMN ONE
%
%%%%%%%%%%%%%%%%%%%%%%%%%%%%%%%%%%%%%%
\begin{minipage}[t]{0.33\textwidth} 

%%%%%%%%%%%%%%%%%%%%%%%%%%%%%%%%%%%%%%
%     EDUCATION
%%%%%%%%%%%%%%%%%%%%%%%%%%%%%%%%%%%%%%
\section{Educação} 
\subsection{Universidad Nacional}
\subsection{de Colombia}
\descript{Engenharia de Sistemas Computacionais}
\location{2016 - 2022  | Bogotá}
\location{Cum. GPA: 4.3 / 5.0}
\sectionsep

%%%%%%%%%%%%%%%%%%%%%%%%%%%%%%%%%%%%%%
%     SOCIAL NETWORKS
%%%%%%%%%%%%%%%%%%%%%%%%%%%%%%%%%%%%%%
\section{Mídia social}
    LinkedIn:// \href{https://www.linkedin.com/in/vanjflores/}{\bf VanJFlorez} \\
    GitHub:// \href{https://github.com/jcammmmm}{\bf VanJFlorez} \\
    GitLab:// \href{https://gitlab.com/VanJFlorez}{\bf VanJFlorez} \\
\sectionsep

%%%%%%%%%%%%%%%%%%%%%%%%%%%%%%%%%%%%%%
%     CERTS
%%%%%%%%%%%%%%%%%%%%%%%%%%%%%%%%%%%%%%
\section{certificados}
-- \textbf{\href{https://jcamilo.co/IELTS-2022.pdf}{Certificado de Inglês B2 (IELTS)}} \\
-- \textbf{\href{https://www.efset.org/cert/j8ebfw}{Certificação de Inglês C1 (EF)}} \\
-- \textbf{\href{https://www.coursera.org/account/accomplishments/certificate/8MS64GGYWDT5?utm_medium=certificate&utm_source=link&utm_campaign=copybutton_certificate}{Machine Learning with Big Data}} \\
-- \textbf{\href{https://www.coursera.org/account/accomplishments/certificate/KFN2XJC2KP92?utm_medium=certificate&utm_source=link&utm_campaign=copybutton_certificate}{Convolutional Neural Networks}} \\
-- \textbf{\href{https://www.udemy.com/certificate/UC-ZHLLMBQA}{Automate Stuff with Python}} \\
-- Econometrics (University) \\
-- Micro and Macroeconomics (University) \\
-- Economic History (University) \\
\smallskip
\scriptsize{Algumas entradas estão vinculadas aos certificados.}
\sectionsep

%{\footnotesize \textit{\textbf{(see github) }}} \\

%%%%%%%%%%%%%%%%%%%%%%%%%%%%%%%%%%%%%%
%     SKILLS
%%%%%%%%%%%%%%%%%%%%%%%%%%%%%%%%%%%%%%
\section{Habilidades}
\locationbold{Linguagens de programação}
    \textbullet{} Java
    \textbullet{} C++
    \textbullet{} JavaScript
    \textbullet{} TypeScript
    \textbullet{} SQL
    \textbullet{} Python
    \textbullet{} GNU/Bash
    \textbullet{} Batch

\locationbold{Estruturas}
    \textbullet{} Spring Boot
    \textbullet{} Quarkus
    \textbullet{} Django
    \textbullet{} EJB JSF JPA (Hibernate) PrimeFaces
    \textbullet{} React/Material-UI
    \textbullet{} Bootstrap
    \textbullet{} Android SDK
    \textbullet{} NumPy
    \textbullet{} Scikit-Learn
    \textbullet{} Pandas
    \textbullet{} Boost C++
    \textbullet{} KNIME
    \textbullet{} Spark

\locationbold{Ferramentas e protocolos}
    \textbullet{} Git
    \textbullet{} Kubernetes
    \textbullet{} Azure
    \textbullet{} Docker
    \textbullet{} ElasticSearch
    \textbullet{} Jaeger
    \textbullet{} OpenShift
    \textbullet{} Azure DevOps
    \textbullet{} Apache Flink
    \textbullet{} Apache Kafka
    \textbullet{} Apache Flume
    \textbullet{} Apache Camel
    \textbullet{} JUnit
    \textbullet{} JMS
    \textbullet{} Mockito
    \textbullet{} Visual Studio
    \textbullet{} Tensorflow
    \textbullet{} SOAP UI
    \textbullet{} GNU-Linux
    \textbullet{} MongoDB
    \textbullet{} Oracle Database 11g
    \textbullet{} SQLite
    \textbullet{} PostgreSQL
    \textbullet{} MySQL
    \textbullet{} Power Designer
    \textbullet{} Virtual Box
    \textbullet{} Linux Servers
    \textbullet{} Audacity
    \textbullet{} Moodle
    \textbullet{} Maven
    \textbullet{} Gradle
    \textbullet{} SSH clients
    \textbullet{} Heroku
    \textbullet{} Jenkins
    \textbullet{} Google Compute Engine
    \textbullet{} TortoiseSVN
    \textbullet{} \LaTeX

\locationbold{Arquiteturas e Metodologias}
    \textbullet{} Microservices
    \textbullet{} REST
    \textbullet{} Web Services
    \textbullet{} SOAP
    \textbullet{} DevOps (Azure)
    \textbullet{} Pipelining
    \textbullet{} Scrum
    \textbullet{} Good Practices (SOLID, Modularity, Design Patterns)


\sectionsep

%%%%%%%%%%%%%%%%%%%%%%%%%%%%%%%%%%%%%%
%
%     COLUMN TWO
%
%%%%%%%%%%%%%%%%%%%%%%%%%%%%%%%%%%%%%%

\end{minipage} 
\hfill
\begin{minipage}[t]{0.66\textwidth} 

%%%%%%%%%%%%%%%%%%%%%%%%%%%%%%%%%%%%%%
%     ABOUT ME
%%%%%%%%%%%%%%%%%%%%%%%%%%%%%%%%%%%%%%

\section{Perfil}
\parindent=20pt Engenheiro de sistemas de computador, desenvolvedor de software e programador de computador com 5 anos de experiência com conhecimento avançado em depuração, matemática, análise de problemas, computação distribuída e design de software. Ele é um aprendiz ágil altamente motivado para atingir metas. Ele tem um excelente domínio de linguagens de programação orientadas a objetos e habilidades atualizadas em técnicas de implementação, criação e implantação de software.

\parindent=20pt Além disso, ele entende com eficiência bibliotecas, estruturas, ferramentas, convenções, decisões de engenharia e os negócios das organizações, pois lê facilmente a documentação de software para uma ampla gama de aplicativos, incluindo aprendizado de máquina, computação gráfica, otimização e matemática aplicada. Com grande sentido de responsabilidade, empenho e bom trabalho em equipa, compromete-se com a missão da organização e da equipa. Ele fala espanhol, inglês e alemão.

% inteligencia artificial %
% análisis de información %
\sectionsep

%%%%%%%%%%%%%%%%%%%%%%%%%%%%%%%%%%%%%%
%     EXPERIENCE
%%%%%%%%%%%%%%%%%%%%%%%%%%%%%%%%%%%%%%
\section{Experiência recente}
    \noindent
    \runsubsection{Teradigit}
    \focusareas{Programação gráfica / Debugging}
    \descript{Software Engineer and Computer Programmer (Self-employed)}
    \location{
        Jun 2023 - Current |
                \includegraphics[width=12pt]{flag-colombia.png}
                Colômbia     }
    \textit{(Tools/Tech. stack) C++, Java11} \\
        - dissecação do código-fonte dos gráficos de computador ProjectM e Blender. \\
        - Depuração do código-fonte Apache Tomcat e RedHat Hibernate. \\
        - implementação de um raytracer baseado em CPU. \\
        \sectionsep

    \noindent
    \runsubsection{Optum}
    \focusareas{Análise de causa raiz / Desenvolvimento e Manutenção de Software / Análise e otimização de SQL}
    \descript{Manager Software Engineer}
    \location{
        Oct 2022 - May 2023 |
                \includegraphics[width=12pt]{flag-usa.png}
                Estados Unidos (remote)    }
    \textit{(Tools/Tech. stack) SQL, Java EE 8, Spring Boot, Java Server Pages, JavaScript, Python} \\
        - Corrigir um relatório financeiro de \textit{40k} USD/ano que fornecia os valores de comissão incorretos. \\
        - Bugs encontrados e corrigidos no aplicativo principal da Optum Financial (\textit{Root Cause Analysis}). \\
        - Análise e otimizações avançadas de SQL. \\
        - Desenvolvimento full stack; de código frontend \textit{SQL DDL/DML} para \textit{JavaScript}. \\
        - Escrever documentação de software. \\
        - Depure e corrija este aplicativo de \textit{15+} anos. \\
        \sectionsep

    \noindent
    \runsubsection{Credera}
    \focusareas{Processamento de Fluxo / Microsserviços}
    \descript{Senior Java Software Developer}
    \location{
        Nov 2021 - Oct 2022 |
                \includegraphics[width=12pt]{flag-usa.png}
                Estados Unidos (remote)    }
    \textit{(Tools/Tech. stack) Kafka, Java 11, AWS, Postgres, Spring Boot, Kubernetes} \\
        - Reimplementação de \textit{monólito} legacy Java7 para \textit{microsserviços} Java11. \\
        - Grave vários produtores-consumidores \textit{Kafka} para os pipelines de streaming de engajamento do usuário. \\
        - Desenvolvimento de microsserviços com \textit{Spring Boot}. \\
        - Desenvolvimento de bibliotecas customizadas para microsserviços e unidades de processamento de ERP que compartilham a mesma lógica de negócio. \\
        - Membro do comitê DE+I (pilar marketing). \\
        \sectionsep

    \noindent
    \runsubsection{Claro}
    \focusareas{Processamento de BigData / Microsserviços}
    \descript{Java Software Developer}
    \location{
        Feb 2020 - Nov 2021 |
                \includegraphics[width=12pt]{flag-brazil.png}
                Brasil (remote)    }
    \textit{(Tools/Tech. stack) Java 11, Apache Flume - Flink - Kafka - Camel, MongoDB, Spring Boot, OpenShift} \\
        - Desenvolvimento de microsserviços com Spring Boot, Quarkus e Django. Alguns exemplos podem ser vistos \textbf{\href{https://github.com/VanJFlorez/flink-kafka-fraud-detection}{aqui}}. \\
        - Componentes de processamento de Big Data usando Apache Kafka, Flink, Flume e Camel. \\
        - Batch Scripts e programas Java para automatizar a criação e o armazenamento de imagens do Docker no registro de nuvem do Docker Hub. \\
        - Implantações em ambientes de pré-produção da Openshift Container Platform. Isso inclui scripts para compilações e testes de estresse de imagem. \\
        - Fornece orientação para desenvolvedores juniores e iniciantes. \\
        - Participe como especialista técnico no projeto de aprimoramentos da interface do usuário 'MiClaro App'. \\
        \sectionsep

    \noindent
    \runsubsection{Zyos}
    \focusareas{Desenvolvedor Back-end Monolith / Desenvolvedor de IU do lado do servidor}
    \descript{FullStack Software Developer}
    \location{
        Jan 2019 - Feb 2020 |
                \includegraphics[width=12pt]{flag-colombia.png}
                Colômbia     }
    \textit{(Tools/Tech. stack) EJB, JSF, Primefaces, JPA (Hibernate), Angular 9, JavaScript (ES6), Maven, Git, Apache TomCat} \\
        - Foram feitas melhorias nos principais recursos do software do servidor \textit{Edificios Davivienda}. \\
        - Desenvolva software \textit{front-end} responsivo e implemente algumas melhorias de estilo \\
        - Implementado de raiz vários componentes \textit{JavaScript} (notificações, avaliações, snackbars) para melhorar a experiência do utilizador. \\
        - Desenvolvimento de software backend para o software servidor \textit{Mi Vivienda BCP} para potencializar e corrigir bugs em casos de execução de ponta. Novas funcionalidades para a aplicação global. \textit{Edificios Davivienda} disponível \textbf{\href{https://www.edificiosdavivienda.com}{aqui}} e \textit{Mi Vivienda BCP} disponível \textbf{\href{https://www.miviviendabcp.com.bo}{aqui}}. \\
        \sectionsep

    \noindent
    \runsubsection{Centro de Pensamiento UNAL}
    \focusareas{}
    \descript{Software Engineer, Requirements Engineer}
    \location{
        Jun 2018 - Dec 2018 |
                Bogotá     }
    \textit{(Tools/Tech. stack) Ruby, RubyOnRails, NGinX, PostgresSQL, Debian-Linux, Moodle, PHP, SSH, Heroku, Google Compute Engine, Git, Docker, Rancker} \\
        - Implementação de curso online com Moodle (disponível \textbf{\href{https://gitlab.com/VanJFlorez/animal-modeling-ethics/tree/master/docs/4 LMS build}{aqui}}) e um aplicativo de fórum da web para esse curso com Thredded (disponível \textbf{\href{https://gitlab.com/VanJFlorez/animal-modeling-ethics/blob/master/docs/3\%20Social\%20WebApp/CP\%20etica\%20animal\%20webApp.pdf}{aqui}}). Repositório completo \textbf{\href{https://gitlab.com/VanJFlorez/animal-modeling-ethics/}{aqui}}. \\
        - Levantamento de requisitos sobre qual solução de software se adapta melhor às necessidades da equipe. \\
        - Personalize as implantações \textit{Moodle} e \textit{Thredded} para atender aos requisitos da equipe. \\
        \sectionsep

    \noindent
    \runsubsection{Grupo Sepro}
    \focusareas{}
    \descript{Software Developer}
    \location{
        Jan 2018 - Jun 2018 |
                Bogotá     }
    \textit{(Tools/Tech. stack) Python, Django, JavaScript, ExpressJS, Gunicorn, Apache HTTP Server, Git, VSCode, MySQL, Heroku} \\
        - Projetou e implementou um sistema de informações de pesquisa que computa estatísticas e medições sobre as respostas enviadas. Este sistema de pesquisa foi usado para avaliar o desempenho das cadeias de fornecimento de derivados de abacate e cana-de-açúcar de Cauca a Bogotá. \\
        - Coordenou uma equipe de 3 alunos de graduação para atingir os objetivos do projeto. Dois deles trabalhavam em desenvolvimento de software e o outro em QA. Disponível \textbf{\href{https://gitlab.com/VanJFlorez/sepro-webapp}{aqui}}. \\
        \sectionsep

 
\vspace{\topsep} % Hacky fix for awkward extra vertical space

%%%%%%%%%%%%%%%%%%%%%%%%%%%%%%%%%%%%%%
%     RESEARCH / PROJECTS
%%%%%%%%%%%%%%%%%%%%%%%%%%%%%%%%%%%%%%

% \section{Proyectos / Investigación}

%\runsubsection{PostConsumeUN - Aplicación Web} \\
%\descript{Proyecto de Curso} 
%\location{Feb 2018 – May 2018}
%Aplicación Web orientada a Microservicios. Usa las tecnologías de Docker, Rancher, RubyOnRails y GraphQL en el servidor. Usa proxy inverso. Disponible \textbf{\href{https://github.com/VanJFlorez/sa_academy_api}{aquí.}} Posee clientes Android (disponible \textbf{\href{https://gitlab.com/VanJFlorez/pcun_ma}{aquí.}} y de navegador web.
%\sectionsep

%\runsubsection{Implementación de algoritmo para tesis doctoral} %\\
%\descript{Asistente de Investigación}
%\location{Nov 2017 – Dec 2017}
%Trabajé con el profesor %\textbf{\href{https://juanmendivelso.com/}{Juan Mendivelso}} %implementando el algoritmo principal de su trabajo de %investigación doctoral ”The Graph Pattern Matching Problem %through Parametrized Matching”. Disponible %\textbf{\href{https://www.microsoft.com/en-us/research/wp-content%/uploads/2016/02/graphisomorphism.spire2013.pdf}{aquí.}}
%\sectionsep

%\runsubsection{Oxygen - Web Application} \\
%\descript{Course Project}
%\location{Aug 2017 – Nov 2017}
%Main project course. Available \textbf{\href{https://github.com/}{Here.}}
%\sectionsep

%%%%%%%%%%%%%%%%%%%%%%%%%%%%%%%%%%%%%%
%     AWARDS
%%%%%%%%%%%%%%%%%%%%%%%%%%%%%%%%%%%%%%

%\section{Awards} 
%\begin{tabular}{rll}
%\end{tabular}
%\sectionsep

%%%%%%%%%%%%%%%%%%%%%%%%%%%%%%%%%%%%%%
%     PUBLICATIONS
%%%%%%%%%%%%%%%%%%%%%%%%%%%%%%%%%%%%%%

%\section{Publications} 
%\renewcommand\refname{\vskip -1.5em} % Couldn't get this working from the .cls file
%\bibliographystyle{abbrv}
%\bibliography{publications}
\nocite{*}

\end{minipage} 
\end{document}  \documentclass[]{article}