%%%%%%%%%%%%%%%%%%%%%%%%%%%%%%%%%%%%%%%
% Deedy - One Page Two Column Resume
% LaTeX Template
% Version 1.8 (13/04/2022)
%
% Original author:
% Debarghya Das ( )
% 
% Devs. and mods.
% Juan Camilo Florez
%
% Original repository:
% https://github.com/deedydas/Deedy-Resume
%
% IMPORTANT: THIS TEMPLATE NEEDS TO BE COMPILED WITH XeLaTeX
%
% This template uses several fonts not included with Windows/Linux by
% default. If you get compilation errors saying a font is missing, find the line
% on which the font is used and either change it to a font included with your
% operating system or comment the line out to use the default font.
% 
%%%%%%%%%%%%%%%%%%%%%%%%%%%%%%%%%%%%%%
% 
% TODO:
% 1. Integrate biber/bibtex for article citation under publications.
% 2. Figure out a smoother way for the document to flow onto the next page.
% 3. Add styling information for a "Projects/Hacks" section.
% 4. Add location/address information
% 5. Merge OpenFont and MacFonts as a single sty with options.
% 
%%%%%%%%%%%%%%%%%%%%%%%%%%%%%%%%%%%%%%
%
% CHANGELOG:
% v1.1:
% 1. Fixed several compilation bugs with \renewcommand
% 2. Got Open-source fonts (Windows/Linux support)
% 3. Added Last Updated
% 4. Move Title styling into .sty
% 5. Commented .sty file.
%
%%%%%%%%%%%%%%%%%%%%%%%%%%%%%%%%%%%%%%%
%
% Known Issues:
% 1. Overflows onto second page if any column's contents are more than the
% vertical limit
% 2. Hacky space on the first bullet point on the second column.
%
%%%%%%%%%%%%%%%%%%%%%%%%%%%%%%%%%%%%%%


\documentclass[]{CV-JuanCamiloFlorez}
\usepackage{fancyhdr}
 
\pagestyle{fancy}
\fancyhf{}
 
\begin{document}

%%%%%%%%%%%%%%%%%%%%%%%%%%%%%%%%%%%%%%
%
%     LAST UPDATED DATE
%
%%%%%%%%%%%%%%%%%%%%%%%%%%%%%%%%%%%%%%
\lastupdated

%%%%%%%%%%%%%%%%%%%%%%%%%%%%%%%%%%%%%%
%
%     TITLE NAME
%
%%%%%%%%%%%%%%%%%%%%%%%%%%%%%%%%%%%%%%
\namesection{Juan Camilo}{Flórez Vanegas}{ \urlstyle{same}
    \href{mailto:jcflorezv@unal.edu.co}{jcflorezv@unal.edu.co} 
    | \href{https://jcamilo.co}{https://jcamilo.co}
    | +57 322 326 67 29
}

%%%%%%%%%%%%%%%%%%%%%%%%%%%%%%%%%%%%%%
%
%     COLUMN ONE
%
%%%%%%%%%%%%%%%%%%%%%%%%%%%%%%%%%%%%%%
\begin{minipage}[t]{0.33\textwidth} 

%%%%%%%%%%%%%%%%%%%%%%%%%%%%%%%%%%%%%%
%     EDUCATION
%%%%%%%%%%%%%%%%%%%%%%%%%%%%%%%%%%%%%%
\section{Ausbildung} 
\subsection{Universidad Nacional}
\subsection{de Colombia}
\descript{Softwaretechnik}
\location{2016 - 2022  | Bogotá}
\location{Cum. GPA: 4.3 / 5.0}
\sectionsep

%%%%%%%%%%%%%%%%%%%%%%%%%%%%%%%%%%%%%%
%     SOCIAL NETWORKS
%%%%%%%%%%%%%%%%%%%%%%%%%%%%%%%%%%%%%%
\section{Sozialen Medien}
    LinkedIn:// \href{https://www.linkedin.com/in/vanjflores/}{\bf VanJFlorez} \\
    GitHub:// \href{https://github.com/jcammmmm}{\bf VanJFlorez} \\
    GitLab:// \href{https://gitlab.com/VanJFlorez}{\bf VanJFlorez} \\
\sectionsep

%%%%%%%%%%%%%%%%%%%%%%%%%%%%%%%%%%%%%%
%     CERTS
%%%%%%%%%%%%%%%%%%%%%%%%%%%%%%%%%%%%%%
\section{Zertifikate}
-- \textbf{\href{https://jcamilo.co/IELTS-2022.pdf}{B2-Englisch-Zertifizierung (IELTS)}} \\
-- \textbf{\href{https://www.efset.org/cert/j8ebfw}{C1-Englisch-Zertifizierung (EF)}} \\
-- \textbf{\href{https://www.coursera.org/account/accomplishments/certificate/8MS64GGYWDT5?utm_medium=certificate&utm_source=link&utm_campaign=copybutton_certificate}{Machine Learning with Big Data}} \\
-- \textbf{\href{https://www.coursera.org/account/accomplishments/certificate/KFN2XJC2KP92?utm_medium=certificate&utm_source=link&utm_campaign=copybutton_certificate}{Convolutional Neural Networks}} \\
-- \textbf{\href{https://www.udemy.com/certificate/UC-ZHLLMBQA}{Automate Stuff with Python}} \\
-- Econometrics (University) \\
-- Micro and Macroeconomics (University) \\
-- Economic History (University) \\
\smallskip
\scriptsize{Einige Einträge sind mit den Zertifikaten verknüpft.}
\sectionsep

%{\footnotesize \textit{\textbf{(see github) }}} \\

%%%%%%%%%%%%%%%%%%%%%%%%%%%%%%%%%%%%%%
%     SKILLS
%%%%%%%%%%%%%%%%%%%%%%%%%%%%%%%%%%%%%%
\section{Fähigkeiten}
\locationbold{Programmiersprachen}
    \textbullet{} Java
    \textbullet{} C++
    \textbullet{} JavaScript
    \textbullet{} TypeScript
    \textbullet{} SQL
    \textbullet{} Python
    \textbullet{} GNU/Bash
    \textbullet{} Charge

\locationbold{Rahmenwerke}
    \textbullet{} Spring Boot
    \textbullet{} EJB JSF JPA (Hibernate) PrimeFaces
    \textbullet{} Quarkus
    \textbullet{} Django
    \textbullet{} React/Material-UI
    \textbullet{} Bootstrap
    \textbullet{} Android SDK
    \textbullet{} NumPy
    \textbullet{} Scikit-Learn
    \textbullet{} Pandas
    \textbullet{} Boost C++
    \textbullet{} KNIME
    \textbullet{} Spark

\locationbold{Tools und Protokolle}
    \textbullet{} Git
    \textbullet{} Kubernetes
    \textbullet{} Azure
    \textbullet{} Docker
    \textbullet{} ElasticSearch
    \textbullet{} Jaeger
    \textbullet{} OpenShift
    \textbullet{} Azure DevOps
    \textbullet{} Apache Flink
    \textbullet{} Apache Kafka
    \textbullet{} Apache Flume
    \textbullet{} Apache Camel
    \textbullet{} JUnit
    \textbullet{} JMS
    \textbullet{} Mockito
    \textbullet{} Visual Studio
    \textbullet{} Tensorflow
    \textbullet{} SOAP UI
    \textbullet{} GNU-Linux
    \textbullet{} MongoDB
    \textbullet{} Oracle Database 11g
    \textbullet{} SQLite
    \textbullet{} PostgreSQL
    \textbullet{} MySQL
    \textbullet{} Power Designer
    \textbullet{} Virtual Box
    \textbullet{} Linux Servers
    \textbullet{} Audacity
    \textbullet{} Moodle
    \textbullet{} Maven
    \textbullet{} Gradle
    \textbullet{} SSH clients
    \textbullet{} Heroku
    \textbullet{} Jenkins
    \textbullet{} Google Compute Engine
    \textbullet{} TortoiseSVN
    \textbullet{} \LaTeX

\locationbold{Architekturen und Methoden}
    \textbullet{} Microservices
    \textbullet{} REST
    \textbullet{} Web Services
    \textbullet{} SOAP
    \textbullet{} DevOps (Azure)
    \textbullet{} Pipelining
    \textbullet{} Scrum
    \textbullet{} Good Practices (SOLID, Modularity, Design Patterns)


\sectionsep

%%%%%%%%%%%%%%%%%%%%%%%%%%%%%%%%%%%%%%
%
%     COLUMN TWO
%
%%%%%%%%%%%%%%%%%%%%%%%%%%%%%%%%%%%%%%

\end{minipage} 
\hfill
\begin{minipage}[t]{0.66\textwidth} 

%%%%%%%%%%%%%%%%%%%%%%%%%%%%%%%%%%%%%%
%     ABOUT ME
%%%%%%%%%%%%%%%%%%%%%%%%%%%%%%%%%%%%%%

\section{Profil}
\parindent=20pt Computersystemingenieur, Softwareentwickler und Computerprogrammierer mit 5 Jahren Erfahrung und fundierten Kenntnissen in den Bereichen Debugging, Mathematik, Problemanalyse, verteiltes Rechnen und Softwaredesign. Er ist ein agiler Lerner, der hoch motiviert ist, Ziele zu erreichen. Er ist mit objektorientierten Programmiersprachen bestens vertraut und verfügt über aktuelle Kenntnisse in Implementierungstechniken, Softwareerstellung und Softwarebereitstellung.

\parindent=20pt Darüber hinaus kann er Programmbibliothek, Frameworks, Tools, Konventionen, technische Entscheidungen und Organisationsunternehmen effizient verstehen, da er problemlos Softwaredokumentationen aus einem breiten Spektrum von Anwendungen liest, darunter maschinelles Lernen, Computergrafik, Optimierung und Angewandte Mathematik. Mit großem Verantwortungsbewusstsein, Engagement und guter Teamarbeit engagiert er sich für Organisation und Teamleitbilder. Er kann Spanisch, Englisch und Portugiesisch.

% inteligencia artificial %
% análisis de información %
\sectionsep

%%%%%%%%%%%%%%%%%%%%%%%%%%%%%%%%%%%%%%
%     EXPERIENCE
%%%%%%%%%%%%%%%%%%%%%%%%%%%%%%%%%%%%%%
\section{Aktuelle Erfahrung}
    \noindent
    \runsubsection{Teradigit}
    \focusareas{Grafikprogrammierung / Debuggen}
    \descript{Software Engineer and Computer Programmer (Self-employed)}
    \location{
        Jun 2023 - Current |
                \includegraphics[width=12pt]{flag-colombia.png}
                Kolumbien     }
    \textit{(Tools/Tech. stack) C++, Java11} \\
        - Zerlegen den Computergrafik-Quellcode von ProjectM und Blender. \\
        - Debuggen den Apache Tomcat und RedHat Hibernate Quellcode. \\
        - Implementieren Sie einen CPU-basierten Raytracer. \\
        \sectionsep

    \noindent
    \runsubsection{Optum}
    \focusareas{Ursachenanalyse / Softwareentwicklung und -wartung / SQL-Analyse und -Optimierung}
    \descript{Manager Software Engineer}
    \location{
        Oct 2022 - May 2023 |
                \includegraphics[width=12pt]{flag-usa.png}
                Vereinigte Staaten (remote)    }
    \textit{(Tools/Tech. stack) SQL, Java EE 8, Spring Boot, Java Server Pages, JavaScript, Python} \\
        - Korrigieren Sie einen Finanzbericht über \textit{40.000} USD/Jahr, der falsche Provisionsbeträge enthielt. \\
        - Fehler in der Optum Financial-Kernanwendung gefunden und behoben (\textit{Root Cause Analysis}). \\
        - Erweiterte SQL-Analyse und -Optimierungen. \\
        - Full-Stack-Entwicklung; von \textit{SQL DDL/DML} zu \textit{JavaScript} Frontend-Code. \\
        - Softwaredokumentation schreiben. \\
        - Debuggen und Patchen dieser über 15 Jahre alten Anwendung. \\
        \sectionsep

    \noindent
    \runsubsection{Credera}
    \focusareas{Stream-Verarbeitung / Microservices}
    \descript{Senior Java Software Developer}
    \location{
        Nov 2021 - Oct 2022 |
                \includegraphics[width=12pt]{flag-usa.png}
                Vereinigte Staaten (remote)    }
    \textit{(Tools/Tech. stack) Kafka, Java 11, AWS, Postgres, Spring Boot, Kubernetes} \\
        - Neuimplementierung von Java7 Legacy \textit{monoliths} zu Java11 \textit{microservices}. \\
        - Schreiben Sie mehrere \textit{Kafka}-Verbraucherproduzenten für die Streaming-Pipelines zur Benutzerinteraktion. \\
        - Microservices-Entwicklung mit \textit{Spring Boot}. \\
        - Entwicklung benutzerdefinierter Bibliotheken für Microservices und ERP-Verarbeitungseinheiten, die dieselbe Geschäftslogik verwenden. \\
        - Mitglied im DE+I-Ausschuss (Säule Marketing). \\
        \sectionsep

    \noindent
    \runsubsection{Claro}
    \focusareas{BigData Processing / Microservices}
    \descript{Java Software Developer}
    \location{
        Feb 2020 - Nov 2021 |
                \includegraphics[width=12pt]{flag-brazil.png}
                Brazil (remote)    }
    \textit{(Tools/Tech. stack) Java 11, Apache Flume - Flink - Kafka - Camel, MongoDB, Spring Boot, OpenShift} \\
        - Microservices-Entwicklung mit Spring Boot, Quarkus und Django. Einige Beispiele sind zu sehen \textbf{\href{https://github.com/VanJFlorez/flink-kafka-fraud-detection}{hier}}. \\
        - Big-Data-Verarbeitungskomponenten mit Apache Kafka, Flink, Flume und Camel. \\
        - Batch-Skripte und Java-Programme zur Automatisierung der Erstellung und Speicherung von Docker-Images in der Docker Hub-Cloud-Registrierung. \\
        - Bereitstellungen in Vorproduktionsumgebungen der Openshift Container Platform. Dazu gehören Skripte für Builds und Image-Stresstests. \\
        - Bieten Sie Mentoring für Junior- und Einsteigerentwickler an. \\
        - Nehmen Sie als technischer Experte am UI-Verbesserungsprojekt „MiClaro App“ teil. \\
        \sectionsep

    \noindent
    \runsubsection{Zyos}
    \focusareas{Backend-Monolith-Entwickler / Serverseitiger UI-Entwickler}
    \descript{FullStack Software Developer}
    \location{
        Jan 2019 - Feb 2020 |
                \includegraphics[width=12pt]{flag-colombia.png}
                Kolumbien     }
    \textit{(Tools/Tech. stack) EJB, JSF, Primefaces, JPA (Hibernate), Angular 9, JavaScript (ES6), Maven, Git, Apache TomCat} \\
        - Es wurden Verbesserungen an den Kernfunktionen der Serversoftware „Edificios Davivienda“ vorgenommen. \\
        - Responsive \textit{Front-End}-Software entwickeln und einige Styling-Verbesserungen implementieren \\
        - Mehrere \textit{JavaScript}-Komponenten (Benachrichtigungen, Bewertungen, Snackbars) wurden von Grund auf implementiert, um die Benutzererfahrung zu verbessern. \\
        - Backend-Softwareentwicklung für \textit{Mi Vivienda BCP}-Serversoftware, um Fehler in Edge-Execution-Fällen zu verstärken und zu korrigieren. Neue Funktionalitäten für die Gesamtanwendung. \textit{Edificios Davivienda} verfügbar \textbf{\href{https://www.edificiosdavivienda.com}{hier}} und \textit{Mi Vivienda BCP} verfügbar \textbf{\href{https://www.miviviendabcp.com.bo}{hier}}. \\
        \sectionsep

    \noindent
    \runsubsection{Centro de Pensamiento UNAL}
    \focusareas{}
    \descript{Software Engineer, Requirements Engineer}
    \location{
        Jun 2018 - Dec 2018 |
                Bogotá     }
    \textit{(Tools/Tech. stack) Ruby, RubyOnRails, NGinX, PostgresSQL, Debian-Linux, Moodle, PHP, SSH, Heroku, Google Compute Engine, Git, Docker, Rancker} \\
        - Online-Kursimplementierung mit Moodle (verfügbar \textbf{\href{https://gitlab.com/VanJFlorez/animal-modeling-ethics/tree/master/docs/4 LMS build}{hier}}) und eine Forum-Webanwendung für diesen Kurs mit Thredded (verfügbar \textbf{\href{https://gitlab.com/VanJFlorez/animal-modeling-ethics/blob/master/docs/3\%20Social\%20WebApp/CP\%20etica\%20animal\%20webApp.pdf}{hier}}). Vollständiges Repository \textbf{\href{https://gitlab.com/VanJFlorez/animal-modeling-ethics/}{hier}}. \\
        - Erfassung der Anforderungen, welche Softwarelösung sich besser an die Teamanforderungen anpasst. \\
        - Passen Sie die \textit{Moodle}- und \textit{Thredded}-Bereitstellungen an, um die Teamanforderungen zu erfüllen. \\
        \sectionsep

    \noindent
    \runsubsection{Grupo Sepro}
    \focusareas{}
    \descript{Software Developer}
    \location{
        Jan 2018 - Jun 2018 |
                Bogotá     }
    \textit{(Tools/Tech. stack) Python, Django, JavaScript, ExpressJS, Gunicorn, Apache HTTP Server, Git, VSCode, MySQL, Heroku} \\
        - Entwarf und implementierte ein Umfrageinformationssystem, das Statistiken und Messungen über die eingereichten Antworten berechnet. Dieses Umfragesystem wurde verwendet, um die Leistung der Lieferketten für Avocado- und Zuckerrohrderivate von Cauca bis Bogota zu bewerten. \\
        - Koordinierte ein Team von drei Bachelor-Studenten, um die Projektziele zu erreichen. Zwei von ihnen arbeiteten in der Softwareentwicklung und der andere in der Qualitätssicherung. Verfügbar \textbf{\href{https://gitlab.com/VanJFlorez/sepro-webapp}{hier}}. \\
        \sectionsep

 
\vspace{\topsep} % Hacky fix for awkward extra vertical space

%%%%%%%%%%%%%%%%%%%%%%%%%%%%%%%%%%%%%%
%     RESEARCH / PROJECTS
%%%%%%%%%%%%%%%%%%%%%%%%%%%%%%%%%%%%%%

% \section{Proyectos / Investigación}

%\runsubsection{PostConsumeUN - Aplicación Web} \\
%\descript{Proyecto de Curso} 
%\location{Feb 2018 – May 2018}
%Aplicación Web orientada a Microservicios. Usa las tecnologías de Docker, Rancher, RubyOnRails y GraphQL en el servidor. Usa proxy inverso. Disponible \textbf{\href{https://github.com/VanJFlorez/sa_academy_api}{aquí.}} Posee clientes Android (disponible \textbf{\href{https://gitlab.com/VanJFlorez/pcun_ma}{aquí.}} y de navegador web.
%\sectionsep

%\runsubsection{Implementación de algoritmo para tesis doctoral} %\\
%\descript{Asistente de Investigación}
%\location{Nov 2017 – Dec 2017}
%Trabajé con el profesor %\textbf{\href{https://juanmendivelso.com/}{Juan Mendivelso}} %implementando el algoritmo principal de su trabajo de %investigación doctoral ”The Graph Pattern Matching Problem %through Parametrized Matching”. Disponible %\textbf{\href{https://www.microsoft.com/en-us/research/wp-content%/uploads/2016/02/graphisomorphism.spire2013.pdf}{aquí.}}
%\sectionsep

%\runsubsection{Oxygen - Web Application} \\
%\descript{Course Project}
%\location{Aug 2017 – Nov 2017}
%Main project course. Available \textbf{\href{https://github.com/}{Here.}}
%\sectionsep

%%%%%%%%%%%%%%%%%%%%%%%%%%%%%%%%%%%%%%
%     AWARDS
%%%%%%%%%%%%%%%%%%%%%%%%%%%%%%%%%%%%%%

%\section{Awards} 
%\begin{tabular}{rll}
%\end{tabular}
%\sectionsep

%%%%%%%%%%%%%%%%%%%%%%%%%%%%%%%%%%%%%%
%     PUBLICATIONS
%%%%%%%%%%%%%%%%%%%%%%%%%%%%%%%%%%%%%%

%\section{Publications} 
%\renewcommand\refname{\vskip -1.5em} % Couldn't get this working from the .cls file
%\bibliographystyle{abbrv}
%\bibliography{publications}
\nocite{*}

\end{minipage} 
\end{document}  \documentclass[]{article}