%%%%%%%%%%%%%%%%%%%%%%%%%%%%%%%%%%%%%%%
% Deedy - One Page Two Column Resume
% LaTeX Template
% Version 1.8 (13/04/2022)
%
% Original author:
% Debarghya Das ( )
% 
% Devs. and mods.
% Juan Camilo Florez
%
% Original repository:
% https://github.com/deedydas/Deedy-Resume
%
% IMPORTANT: THIS TEMPLATE NEEDS TO BE COMPILED WITH XeLaTeX
%
% This template uses several fonts not included with Windows/Linux by
% default. If you get compilation errors saying a font is missing, find the line
% on which the font is used and either change it to a font included with your
% operating system or comment the line out to use the default font.
% 
%%%%%%%%%%%%%%%%%%%%%%%%%%%%%%%%%%%%%%
% 
% TODO:
% 1. Integrate biber/bibtex for article citation under publications.
% 2. Figure out a smoother way for the document to flow onto the next page.
% 3. Add styling information for a "Projects/Hacks" section.
% 4. Add location/address information
% 5. Merge OpenFont and MacFonts as a single sty with options.
% 
%%%%%%%%%%%%%%%%%%%%%%%%%%%%%%%%%%%%%%
%
% CHANGELOG:
% v1.1:
% 1. Fixed several compilation bugs with \renewcommand
% 2. Got Open-source fonts (Windows/Linux support)
% 3. Added Last Updated
% 4. Move Title styling into .sty
% 5. Commented .sty file.
%
%%%%%%%%%%%%%%%%%%%%%%%%%%%%%%%%%%%%%%%
%
% Known Issues:
% 1. Overflows onto second page if any column's contents are more than the
% vertical limit
% 2. Hacky space on the first bullet point on the second column.
%
%%%%%%%%%%%%%%%%%%%%%%%%%%%%%%%%%%%%%%


\documentclass[]{CV-JuanCamiloFlorez}
\usepackage{fancyhdr}
 
\pagestyle{fancy}
\fancyhf{}
 
\begin{document}

%%%%%%%%%%%%%%%%%%%%%%%%%%%%%%%%%%%%%%
%
%     LAST UPDATED DATE
%
%%%%%%%%%%%%%%%%%%%%%%%%%%%%%%%%%%%%%%
\lastupdated

%%%%%%%%%%%%%%%%%%%%%%%%%%%%%%%%%%%%%%
%
%     TITLE NAME
%
%%%%%%%%%%%%%%%%%%%%%%%%%%%%%%%%%%%%%%
\namesection{Juan Camilo}{Flórez Vanegas}{ \urlstyle{same}
    \href{mailto:jcflorezv@unal.edu.co}{jcflorezv@unal.edu.co} 
    | \href{https://jcamilo.co}{https://jcamilo.co}
    | +57 322 326 67 29
}

%%%%%%%%%%%%%%%%%%%%%%%%%%%%%%%%%%%%%%
%
%     COLUMN ONE
%
%%%%%%%%%%%%%%%%%%%%%%%%%%%%%%%%%%%%%%
\begin{minipage}[t]{0.33\textwidth} 

%%%%%%%%%%%%%%%%%%%%%%%%%%%%%%%%%%%%%%
%     EDUCATION
%%%%%%%%%%%%%%%%%%%%%%%%%%%%%%%%%%%%%%
\section{Educación} 
\subsection{Universidad Nacional}
\subsection{de Colombia}
\descript{Ingeniería de Sistemas y Computación}
\location{2016 - 2022  | Bogotá}
\location{Cum. GPA: 4.3 / 5.0}
\sectionsep

%%%%%%%%%%%%%%%%%%%%%%%%%%%%%%%%%%%%%%
%     SOCIAL NETWORKS
%%%%%%%%%%%%%%%%%%%%%%%%%%%%%%%%%%%%%%
\section{Redes sociales}
    LinkedIn:// \href{https://www.linkedin.com/in/vanjflores/}{\bf VanJFlorez} \\
    GitHub:// \href{https://github.com/jcammmmm}{\bf VanJFlorez} \\
    GitLab:// \href{https://gitlab.com/VanJFlorez}{\bf VanJFlorez} \\
\sectionsep

%%%%%%%%%%%%%%%%%%%%%%%%%%%%%%%%%%%%%%
%     CERTS
%%%%%%%%%%%%%%%%%%%%%%%%%%%%%%%%%%%%%%
\section{Certificados}
-- \textbf{\href{https://jcamilo.co/IELTS-2022.pdf}{Certificación Inglés B2 (IELTS)}} \\
-- \textbf{\href{https://www.efset.org/cert/j8ebfw}{Certificación Inglés C1 (EF)}} \\
-- \textbf{\href{https://www.coursera.org/account/accomplishments/certificate/8MS64GGYWDT5?utm_medium=certificate&utm_source=link&utm_campaign=copybutton_certificate}{Machine Learning with Big Data}} \\
-- \textbf{\href{https://www.coursera.org/account/accomplishments/certificate/KFN2XJC2KP92?utm_medium=certificate&utm_source=link&utm_campaign=copybutton_certificate}{Convolutional Neural Networks}} \\
-- \textbf{\href{https://www.udemy.com/certificate/UC-ZHLLMBQA}{Automate Stuff with Python}} \\
-- Econometría (Universidad) \\
-- Micro y macroeconomía (Universidad) \\
-- Historia económica (Universidad) \\
\smallskip
\scriptsize{}
\sectionsep

%{\footnotesize \textit{\textbf{(see github) }}} \\

%%%%%%%%%%%%%%%%%%%%%%%%%%%%%%%%%%%%%%
%     SKILLS
%%%%%%%%%%%%%%%%%%%%%%%%%%%%%%%%%%%%%%
\section{Habilidades}
\locationbold{Lenguajes de programación}
    \textbullet{} Java
    \textbullet{} C++
    \textbullet{} JavaScript
    \textbullet{} TypeScript
    \textbullet{} SQL
    \textbullet{} Python
    \textbullet{} GNU/Bash
    \textbullet{} Batch

\locationbold{Frameworks}
    \textbullet{} Spring Boot
    \textbullet{} Quarkus
    \textbullet{} Django
    \textbullet{} EJB JSF JPA (Hibernate) PrimeFaces
    \textbullet{} React/Material-UI
    \textbullet{} Bootstrap
    \textbullet{} Android SDK
    \textbullet{} NumPy
    \textbullet{} Scikit-Learn
    \textbullet{} Pandas
    \textbullet{} Boost C++
    \textbullet{} KNIME
    \textbullet{} Spark

\locationbold{Herramientas y protocolos}
    \textbullet{} Git
    \textbullet{} Kubernetes
    \textbullet{} Azure
    \textbullet{} Docker
    \textbullet{} ElasticSearch
    \textbullet{} Jaeger
    \textbullet{} OpenShift
    \textbullet{} Azure DevOps
    \textbullet{} Apache Flink
    \textbullet{} Apache Kafka
    \textbullet{} Apache Flume
    \textbullet{} Apache Camel
    \textbullet{} JUnit
    \textbullet{} JMS
    \textbullet{} Mockito
    \textbullet{} Visual Studio
    \textbullet{} Tensorflow
    \textbullet{} SOAP UI
    \textbullet{} GNU-Linux
    \textbullet{} MongoDB
    \textbullet{} Oracle Database 11g
    \textbullet{} SQLite
    \textbullet{} PostgreSQL
    \textbullet{} MySQL
    \textbullet{} Power Designer
    \textbullet{} Virtual Box
    \textbullet{} Linux Servers
    \textbullet{} Audacity
    \textbullet{} Moodle
    \textbullet{} Maven
    \textbullet{} Gradle
    \textbullet{} SSH clients
    \textbullet{} Heroku
    \textbullet{} Jenkins
    \textbullet{} Google Compute Engine
    \textbullet{} TortoiseSVN
    \textbullet{} \LaTeX

\locationbold{Arquitecturas y metodologías}
    \textbullet{} Microservices
    \textbullet{} REST
    \textbullet{} Web Services
    \textbullet{} SOAP
    \textbullet{} DevOps (Azure)
    \textbullet{} Pipelining
    \textbullet{} Scrum
    \textbullet{} Good Practices (SOLID, Modularity, Design Patterns)


\sectionsep

%%%%%%%%%%%%%%%%%%%%%%%%%%%%%%%%%%%%%%
%
%     COLUMN TWO
%
%%%%%%%%%%%%%%%%%%%%%%%%%%%%%%%%%%%%%%

\end{minipage} 
\hfill
\begin{minipage}[t]{0.66\textwidth} 

%%%%%%%%%%%%%%%%%%%%%%%%%%%%%%%%%%%%%%
%     ABOUT ME
%%%%%%%%%%%%%%%%%%%%%%%%%%%%%%%%%%%%%%

\section{Perfíl}
\parindent=20pt Ingeniero en sistemas computacionales, desarrollador de software y programador de computadoras con 5 años de experiencia con conocimientos avanzados en depuración, matemáticas, análisis de problemas, computación distribuida y diseño de software. Es un aprendiz ágil altamente motivado por alcanzar metas. Posee un excelente dominio de lenguajes de programación orientados a objetos y habilidades actualizadas en técnicas de implementación, creación y despliegue de software.

\parindent=20pt Además, comprende de manera eficiente librerías, frameworks, herramientas, convenciones, decisiones de ingeniería y los negocios de organizaciones, ya que lee con facilidad documentación de software de una amplia gama de aplicaciones que incluyen aprendizaje de máquina, gráficos por computadora, optimización y matemáticas aplicadas. Con un gran sentido de la responsabilidad, compromiso y buen trabajo en equipo, se compromete con la misión de la organización y el equipo. Puede hablar inglés, portugués y alemán.

% inteligencia artificial %
% análisis de información %
\sectionsep

%%%%%%%%%%%%%%%%%%%%%%%%%%%%%%%%%%%%%%
%     EXPERIENCE
%%%%%%%%%%%%%%%%%%%%%%%%%%%%%%%%%%%%%%
\section{Experiencia reciente}
    \noindent
    \runsubsection{Teradigit}
    \focusareas{Programación de gráficos por computadora / Debugging}
    \descript{Ingeniero de Software y programador de computadoras (Self-employed)}
    \location{
        Jun 2023 - Current |
                \includegraphics[width=12pt]{flag-colombia.png}
                Colombia     }
    \textit{(Tools/Tech. stack) C++, Java11} \\
        - disección del código fuente de gráficos de computadora de ProjectM y Blender. \\
        - depuración del código fuente de Apache Tomcat y RedHat Hibernate. \\
        - implentación de un raytracer basado en CPU. \\
        \sectionsep

    \noindent
    \runsubsection{Optum}
    \focusareas{Análisis de causa raíz / Desarrollo y Mantenimiento de Software / Análisis y optimización de SQL}
    \descript{Manager Software Engineer}
    \location{
        Oct 2022 - May 2023 |
                \includegraphics[width=12pt]{flag-usa.png}
                Estados Unidos (remote)    }
    \textit{(Tools/Tech. stack) SQL, Java EE 8, Spring Boot, Java Server Pages, JavaScript, Python} \\
        - Arreglar un informe financiero de \textit{40k} USD/año que proporcionaba montos de comisión incorrectos. \\
        - se encontraron y corrigieron errores en la aplicación principal de Optum Financial (\textit{Análisis de causa raíz}). \\
        - Análisis y optimizaciones avanzadas de SQL. \\
        - desarrollo de pila completa; desde \textit{SQL DDL/DML} a \textit{JavaScript} código frontend. \\
        - Escribir la documentación del software. \\
        - Depurar y parcher esta aplicación de \textit{15+} años. \\
        \sectionsep

    \noindent
    \runsubsection{Credera}
    \focusareas{Stream Processing / Microservices}
    \descript{Senior Java Software Developer}
    \location{
        Nov 2021 - Oct 2022 |
                \includegraphics[width=12pt]{flag-usa.png}
                United States (remote)    }
    \textit{(Tools/Tech. stack) Kafka, Java 11, AWS, Postgres, Spring Boot, Kubernetes} \\
        - Reimplementación de \textit{monolito} escrito en Java7 a \textit{microservicios} en Java11. \\
        - Implementación de varios consumidores-productores \textit{Kafka} para las canalizaciones de transmisión de participación del usuario. \\
        - Desarrollo de microservicios con \textit{Spring Boot}. \\
        - Desarrollo de librerías personalizadas para microservicios y unidades de procesamiento ERP que comparten la misma lógica de negocio. \\
        - Miembro del comité DE+I (pilar de marketing). \\
        \sectionsep

    \noindent
    \runsubsection{Claro}
    \focusareas{BigData Processing / Microservices}
    \descript{Java Software Developer}
    \location{
        Feb 2020 - Nov 2021 |
                \includegraphics[width=12pt]{flag-brazil.png}
                Brazil (remote)    }
    \textit{(Tools/Tech. stack) Java 11, Apache Flume - Flink - Kafka - Camel, MongoDB, Spring Boot, OpenShift} \\
        - Desarrollo de microservicios con Spring Boot, Quarkus y Django. Se pueden ver algunos ejemplos \textbf{\href{https://github.com/VanJFlorez/flink-kafka-fraud-detection}{aquí}}. \\
        - Componentes de procesamiento de Big Data utilizando Apache Kafka, Flink, Flume y Camel. \\
        - Batch Scripts y programas Java para automatizar la compilación y el almacenamiento de imágenes de Docker dentro del registro en la nube de Docker Hub. \\
        - Implementaciones en entornos de preproducción de Openshift Container Platform. Esto incluye scripts para compilaciones y pruebas de estrés de imágenes. \\
        - Proporcionar tutoría para desarrolladores junior y de nivel de entrada. \\
        - Participé como experto técnico en el proyecto de mejoras de la interfaz de usuario de la aplicación 'MiClaro'. \\
        \sectionsep

    \noindent
    \runsubsection{Zyos}
    \focusareas{Desarrollador de monolito (backend) / Desarrollador de interfaz UI en servidor}
    \descript{FullStack Software Developer}
    \location{
        Jan 2019 - Feb 2020 |
                \includegraphics[width=12pt]{flag-colombia.png}
                Colombia     }
    \textit{(Tools/Tech. stack) EJB, JSF, Primefaces, JPA (Hibernate), Angular 9, JavaScript (ES6), Maven, Git, Apache TomCat} \\
        - Se realizaron mejoras en las características principales del software del servidor \textit{Edificios Davivienda}. \\
        - Desarrolle un software \textit{front-end} receptivo e implemente algunas mejoras de estilo \\
        - Implementé desde cero varios componentes de \textit{JavaScript} (notificaciones, calificaciones, nackbars) para mejorar la experiencia del usuario. \\
        - Desarrollo de software backend para el software del servidor \textit{Mi Vivienda BCP} con el fin de potenciar y corregir errores en los casos de ejecución complicados. Nuevas funcionalidades a la aplicación en general. \textit{Edificios Davivienda} disponible \textbf{\href{https://www.edificiosdavivienda.com}{aquí}} y \textit{Mi Vivienda BCP} disponible \textbf{\href{https://www.miviviendabcp.com.bo}{aquí}}. \\
        \sectionsep

    \noindent
    \runsubsection{Centro de Pensamiento UNAL}
    \focusareas{}
    \descript{Software Engineer, Requirements Engineer}
    \location{
        Jun 2018 - Dec 2018 |
                Bogotá     }
    \textit{(Tools/Tech. stack) Ruby, RubyOnRails, NGinX, PostgresSQL, Debian-Linux, Moodle, PHP, SSH, Heroku, Google Compute Engine, Git, Docker, Rancker} \\
        - Implementación de un curso en línea con Moodle (disponible \textbf{\href{https://gitlab.com/VanJFlorez/animal-modeling-ethics/tree/master/docs/4 LMS build}{aquí}}) y una aplicación web de foro para ese curso con Thredded (disponible \textbf{\href{https://gitlab.com/VanJFlorez/animal-modeling-ethics/blob/master/docs/3\%20Social\%20WebApp/CP\%20etica\%20animal\%20webApp.pdf}{aquí}}). Repositorio completo \textbf{\href{https://gitlab.com/VanJFlorez/animal-modeling-ethics/}{aquí}}. \\
        - Levantamiento de requerimientos sobre qué solución de software se adapta mejor a las necesidades del equipo. \\
        - Personalicé las instalaciones de \textit{Moodle} y \textit{Thredded} para cumplir con los requisitos del equipo. \\
        \sectionsep

    \noindent
    \runsubsection{Grupo Sepro}
    \focusareas{}
    \descript{Software Developer}
    \location{
        Jan 2018 - Jun 2018 |
                Bogotá     }
    \textit{(Tools/Tech. stack) Python, Django, JavaScript, ExpressJS, Gunicorn, Apache HTTP Server, Git, VSCode, MySQL, Heroku} \\
        - Diseno e implementación un sistema de información de encuestas que calcula estadísticas y mediciones sobre las respuestas enviadas. Este sistema de encuestas se utilizó para evaluar el desempeño de las cadenas de suministro de derivados del aguacate y la caña de azúcar desde el Cauca hasta Bogotá. \\
        - Coordiné un equipo de 3 estudiantes de pregrado para lograr los objetivos del proyecto. Dos de ellos trabajaban en desarrollo de software y el otro en control de calidad. Disponible \textbf{\href{https://gitlab.com/VanJFlorez/sepro-webapp}{aquí}}. \\
        \sectionsep

 
\vspace{\topsep} % Hacky fix for awkward extra vertical space

%%%%%%%%%%%%%%%%%%%%%%%%%%%%%%%%%%%%%%
%     RESEARCH / PROJECTS
%%%%%%%%%%%%%%%%%%%%%%%%%%%%%%%%%%%%%%

% \section{Proyectos / Investigación}

%\runsubsection{PostConsumeUN - Aplicación Web} \\
%\descript{Proyecto de Curso} 
%\location{Feb 2018 – May 2018}
%Aplicación Web orientada a Microservicios. Usa las tecnologías de Docker, Rancher, RubyOnRails y GraphQL en el servidor. Usa proxy inverso. Disponible \textbf{\href{https://github.com/VanJFlorez/sa_academy_api}{aquí.}} Posee clientes Android (disponible \textbf{\href{https://gitlab.com/VanJFlorez/pcun_ma}{aquí.}} y de navegador web.
%\sectionsep

%\runsubsection{Implementación de algoritmo para tesis doctoral} %\\
%\descript{Asistente de Investigación}
%\location{Nov 2017 – Dec 2017}
%Trabajé con el profesor %\textbf{\href{https://juanmendivelso.com/}{Juan Mendivelso}} %implementando el algoritmo principal de su trabajo de %investigación doctoral ”The Graph Pattern Matching Problem %through Parametrized Matching”. Disponible %\textbf{\href{https://www.microsoft.com/en-us/research/wp-content%/uploads/2016/02/graphisomorphism.spire2013.pdf}{aquí.}}
%\sectionsep

%\runsubsection{Oxygen - Web Application} \\
%\descript{Course Project}
%\location{Aug 2017 – Nov 2017}
%Main project course. Available \textbf{\href{https://github.com/}{Here.}}
%\sectionsep

%%%%%%%%%%%%%%%%%%%%%%%%%%%%%%%%%%%%%%
%     AWARDS
%%%%%%%%%%%%%%%%%%%%%%%%%%%%%%%%%%%%%%

%\section{Awards} 
%\begin{tabular}{rll}
%\end{tabular}
%\sectionsep

%%%%%%%%%%%%%%%%%%%%%%%%%%%%%%%%%%%%%%
%     PUBLICATIONS
%%%%%%%%%%%%%%%%%%%%%%%%%%%%%%%%%%%%%%

%\section{Publications} 
%\renewcommand\refname{\vskip -1.5em} % Couldn't get this working from the .cls file
%\bibliographystyle{abbrv}
%\bibliography{publications}
\nocite{*}

\end{minipage} 
\end{document}  \documentclass[]{article}